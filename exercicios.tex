\documentclass[10pt]{article}

\usepackage[utf8]{inputenc}
\usepackage[portuguese]{babel}
\usepackage{auto-pst-pdf}
\usepackage{fullpage}
\usepackage{latexsym}
\usepackage{vaucanson-g}
\usepackage{amsmath}


\newcommand{\quest}[1] {\vspace{0.5cm}\noindent {\bf {#1})}\\}
\newcommand{\partQuest}[1] {\noindent {\bf {#1})}}
\newcommand{\fimprova}{\begin{flushright}$\Box$\end{flushright}}

%%%%%%%%%%%%%%%%%%%%%%%%%%%%%%%%%%%%%%%%%%%%%%%%%%%%%
%
%
%			Título
%
%
%%%%%%%%%%%%%%%%%%%%%%%%%%%%%%%%%%%%%%%%%%%%%%%%%%%%%
\title{ {\footnotesize
	\hrule\vspace{1pt}\hrule\vspace{1ex}
		Instituto de Computação \hfill Universidade Estadual de Campinas
	\smallskip 
	\hrule\vspace{1pt}\hrule}\vspace{10pt}
		MO405 --- Teoria de Grafos I \\[-6pt]
	\author{Jefferson e Rafael} 
}

\date{\bf Primeiro Semestre de 2011}

%%%%%%%%%%%%%%%%%%%%%%%%%%%%%%%%%%%%%%%%%%%%%%%%%%%%%
%
%
%			Soluçőes
%
%
%%%%%%%%%%%%%%%%%%%%%%%%%%%%%%%%%%%%%%%%%%%%%%%%%%%%%
\begin{document}
 
\maketitle
\vspace{0.5cm}
\thispagestyle{empty}


% Questőes
%\quest{1.1.2}

Seja $G$ um grafo simples com $n$ vértices numerados de $1$ até $n$.
%
Considere a matriz de adjacências {\bf A}$_{n,n}$ associada a $G$.
%
Neste caso a matriz é definida como $${\bf A}_{i,j} = \left\{ \begin{array}{lr} 1, & se\quad\{i,j\} \in E(G) \\ 0, & c. c. \end{array}\right.$$
%
Como $G$ é simples $A_{i,i}$, $1 \le i \le n$ e, portanto, $\sum_{1 \le i \neq j \le n} A_{i_j} \le n\cdot (n-1)$.
%
Pelo teorema fundamental temos que 
\begin{eqnarray}
	2m &=& \sum_{1 \le i \neq j \le n} A_{i_j} \le n\cdot(n-1) \nonumber\\
	m  &\le& \frac{n\cdot(n-1)}{2} = \left(\begin{array}{c} n \\ 2 \end{array}\right) \nonumber
\end{eqnarray}
\fimprova

Além disto, quando $G$ é completo ($K_n$) temos que a matriz {\bf A} fica definida como $${\bf A}_{i,j} = \left\{ \begin{array}{lr} 1, & se\quad ssi\neq j \\ 0, & c. c. \end{array}\right.$$ logo
\begin{eqnarray}
	2m &=& \sum_{1 \le i \neq j \le n} A_{i_j} = n\cdot(n-1) \nonumber\\
	m  &=& \frac{n\cdot(n-1)}{2} = \left(\begin{array}{c} n \\ 2 \end{array}\right) \nonumber
\end{eqnarray}
\fimprova

%\quest{1.1.3}

\partQuest{a} Seja $G[X,Y]$ um grafo simples bipartido, onde $|X| = r$ e $|Y| = s$, considere a matriz de adjacências {\bf A}$_{r,s}$ associada a $G$.
%
Neste caso a matriz é definida como $${\bf A}_{i,j} = \left\{ \begin{array}{lr} 1, & se\quad\{i,j\} \in E(G) \\ 0, & c. c. \end{array}\right.$$
%
Portanto, como não há vértices repetidos nas linhas e colunas da matriz {\bf A} temos que $$m = \sum_{\substack{1 \le i \le r \\ 1 \le j \le s}} A_{ij} \le rs$$.
\fimprova

\partQuest{b} Basta mostrar o máximo da função $f(r,s) = r.s$, onde $1 \le r \le n$, $1 \le s \le n$ e $r = s$.
%
Para isto buscamos os pontos críticos de $f$ com: 
\begin{eqnarray}
	\frac{\partial f}{\partial r} = 0 &\Rightarrow& s = 0 \nonumber\\
	\frac{\partial f}{\partial s} = 0 &\Rightarrow& r = 0 \nonumber
\end{eqnarray}
Como nenhum dos pontos pertence ao domínio, devemos analisar os pontos extremais nos limites do domínio, em particular, quando $r = s$.
%
Neste caso, $f(r,s) = r(n - r) = nr - r^2$, onde $n$ é constante e $1 \le r \le n$.
%
Buscando os pontos de mínimo temos: $$\frac{\partial f}{\partial r} = 0 \Rightarrow n - 2r = 0 \Rightarrow r = \frac{n}{2}$$

Além disto, é $n = \frac{r}{2}$ é ponto de máximo pois: $$\frac{\partial f^2}{\partial r\partial r} = -2 < 0$$.

Portanto, o máximo da função $f$ ocorre quando $r = s = \frac{n}{2}$ com valor $\frac{n^2}{4}$.
\fimprova

\partQuest{b} Os grafos bipartidos completos $K_{\frac{n}{2},\frac{n}{2}}$ possuem $n$ vértices e, exatamente $\frac{n^2}{4}$ arestas.


\quest{4.1.20}

\partQuest{a}
($\Rightarrow$) Suponha por absurdo que $T_1 \cap T_2 = \varnothing$.
%
Se $u \in V(T_1)$ e $v \in V(T_2)$ então não existe caminho de $u$ para $v$ em $T_1 \cup T_2$, pois não há vertices em comum entre $T_1$ e $T_2$.
%
Mas isso contraria a hipótese de que $T_1 \cup T_2$ é uma subárvore de $T$.
%
Logo, $T_1 \cap T_2 \ne \varnothing$.

($\Leftarrow$) Seja $T_1$ e $T_2$ subárvores de $T$, tal que $T_1 \cap T_2 \ne \varnothing$. 
%
Obviamente $T_1 \cap T_2$ é acíclico, pois senão ambas árvores teriam ciclos e, além disto, $T_1 \cap T_2$ é conexo, pois se $u,v \in V(T_1 \cap T_2)$, então $u,v \in V(T_1)$ e, por definição $\exists uT_1v$.
%
Temos também que $T_1 \cup T_2$ é um grafo acíclico, senão $T$ conteria um ciclo, o que é um absurdo pois $T$ é árvore.
%
Resta-nos mostrar que $T_1 \cup T_2$ é conexo, e,  para tanto, considere $u,v \in V(T_1 \cup T_2)$.
%
Se $u,v \in V(T_1)$ ou $u,v \in V(T_2)$, temos, por definição, que existe caminho entre $u$ e $v$.
%
Sem perda de generalidade, considere que $u \in V(T_1)$ e $v \in V(T_2)$.
%
Seja $y \in T_1 \cap T_2$, então $\exists uT_1y$ e $\exists yT_2v$, logo, $\exists uT_1yT_2v$ em $T_1 \cup T_2$.
\fimprova

\partQuest{b}
Prova por indução forte em $|{\cal T}| = n$.

Hipótese: Se ${\cal T}$ é uma família de subárvores de uma árvore $T$ com
$|{\cal T}| = k$, $ 1 \le k \le n$, se quaisquer 2 membros de ${\cal T}$ possuem
um vértice em comum, então há um vértice de $T$ que pertence a todos os membros
de ${\cal T}$.

Passo: Seja ${\cal T}$ uma família de subárvores da árvore $T$ como na hipótese e
tal que $|{\cal T}| = n + 1$. Sejam $T'$ e $T''$ subárvores de ${\cal T}$ tal que
$T' \cap T'' \ne \varnothing$. Sabemos de (a) que $T' \cap T''$ é subárvore de $T$.
Seja ${\cal T'} = {\cal T} - \{T', T''\} \cup \{T' \cap T''\}$ a família obtida
por substituir $T'$ e $T''$ por sua intersecção em ${\cal T}$. Em ${\cal T'}$,
temos que $T_i \cap T_j \ne \varnothing$, $\forall T_i, T_j$. Então, por H.I.
existe um vértice $v$ de $T$ que pertence a todos os membros de ${\cal T'}$. Em
particular, $v$ pertence a $T' \cap T''$ e, portanto, pertence a $T'$ e $T''$.
\fimprova


% Exemplos
\quest{???} Exemplos de grafos$\ldots$

\vspace{0.2cm}


%%%%%%%%%%%%%%%%%%%%%%%%%%%%%%%%%%%%%%%%%%%%%%%%%%%%%
%
%
%			Petersen
%
%
%%%%%%%%%%%%%%%%%%%%%%%%%%%%%%%%%%%%%%%%%%%%%%%%%%%%%
\begin{figure} [htb]
	\centering
	\begin{postscript}
		\TinyPicture\VCDraw{%
		\begin{VCPicture}{(0,0)(10,9)}
			\ChgEdgeArrowStyle{-}
			%%%%%%%%%%%%%%%%%%%%%%%%%%%% vertices %%%%%%%%%%%%%%%%%%%%%%%%%
			% Anel externo
			\State[v_1]{(5,9)}{1} \State[v_2]{(10,5)}{2} \State[v_3]{(8,0)}{3} \State[v_4]{(2,0)}{4} \State[v_5]{(0,5)}{5} 
			% Anel interno
			\State[v_6]{(5,7)}{6} \State[v_7]{(7,5)}{7} \State[v_8]{(6,2)}{8} \State[v_9]{(4,2)}{9} \State[v_{10}]{(3,5)}{10}
			%%%%%%%%%%%%%%%%%%%%%%%%%%%% Arestas %%%%%%%%%%%%%%%%%%%%%%%%%			
			\EdgeL{1}{2}{} \EdgeL{2}{3}{} \EdgeL{3}{4}{} \EdgeL{4}{5}{} \EdgeL{5}{1}{}
			\EdgeL{1}{6}{} \EdgeL{2}{7}{} \EdgeL{3}{8}{} \EdgeL{4}{9}{} \EdgeL{5}{10}{}
			\EdgeL{6}{8}{} \EdgeR[.1]{6}{9}{} \EdgeL{7}{9}{} \EdgeR{7}{10}{} \EdgeL{8}{10}{}
			%
		\end{VCPicture}}
	\end{postscript}
	\caption {Petersen Graph.}
\end{figure}


%%%%%%%%%%%%%%%%%%%%%%%%%%%%%%%%%%%%%%%%%%%%%%%%%%%%%
%
%
%		O "NOSSO AMIGO"???
%
%
%%%%%%%%%%%%%%%%%%%%%%%%%%%%%%%%%%%%%%%%%%%%%%%%%%%%%
\begin{figure} [htb]
	\centering
	\begin{postscript}
		\TinyPicture\VCDraw{%
		\begin{VCPicture}{(0,-2)(6,2)}
			\ChgEdgeArrowStyle{-}
			%%%%%%%%%%%%%%%%%%%%%%%%%%%% vertices %%%%%%%%%%%%%%%%%%%%%%%%%
			\State[v_1]{(0,6)}{1} \State[v_2]{(10,6)}{2} \State[v_3]{(3,3)}{3} \State[v_4]{(7,3)}{4} \State[v_5]{(0,0)}{5} \State[v_6]{(10,0)}{6}
			% Edges
			\EdgeL{1}{2}{e_1} \EdgeL{1}{3}{e_2} \EdgeL{1}{5}{e_3} \EdgeL{3}{5}{e_4} \EdgeL{3}{4}{e_5}
			\EdgeL{5}{6}{e_6} \EdgeL{2}{4}{e_7} \EdgeL{4}{6}{e_8} \EdgeL{2}{6}{e_9}
			%
		\end{VCPicture}}
	\end{postscript}
	\caption {O Nosso amigo $\overline{C_6}$.}
\end{figure}


\end{document}



