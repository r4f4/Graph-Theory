\quest{1.1.3}

\partQuest{a} Seja $G[X,Y]$ um grafo simples bipartido, onde $|X| = r$ e $|Y| = s$, considere a matriz de adjacências {\bf A}$_{r,s}$ associada a $G$.
%
Neste caso a matriz é definida como $${\bf A}_{i,j} = \left\{ \begin{array}{lr} 1, & se\quad\{i,j\} \in E(G) \\ 0, & c. c. \end{array}\right.$$
%
Portanto, como não há vértices repetidos nas linhas e colunas da matriz {\bf A} temos que $$m = \sum_{\substack{1 \le i \le r \\ 1 \le j \le s}} A_{ij} \le rs$$.
\fimprova

\partQuest{b} Basta mostrar o máximo da função $f(r,s) = r.s$, onde $1 \le r \le n$, $1 \le s \le n$ e $r = s$.
%
Para isto buscamos os pontos críticos de $f$ com: 
\begin{eqnarray}
	\frac{\partial f}{\partial r} = 0 &\Rightarrow& s = 0 \nonumber\\
	\frac{\partial f}{\partial s} = 0 &\Rightarrow& r = 0 \nonumber
\end{eqnarray}
Como nenhum dos pontos pertence ao domínio, devemos analisar os pontos extremais nos limites do domínio, em particular, quando $r = s$.
%
Neste caso, $f(r,s) = r(n - r) = nr - r^2$, onde $n$ é constante e $1 \le r \le n$.
%
Buscando os pontos de mínimo temos: $$\frac{\partial f}{\partial r} = 0 \Rightarrow n - 2r = 0 \Rightarrow r = \frac{n}{2}$$

Além disto, é $n = \frac{r}{2}$ é ponto de máximo pois: $$\frac{\partial f^2}{\partial r\partial r} = -2 < 0$$.

Portanto, o máximo da função $f$ ocorre quando $r = s = \frac{n}{2}$ com valor $\frac{n^2}{4}$.
\fimprova

\partQuest{b} Os grafos bipartidos completos $K_{\frac{n}{2},\frac{n}{2}}$ possuem $n$ vértices e, exatamente $\frac{n^2}{4}$ arestas.

