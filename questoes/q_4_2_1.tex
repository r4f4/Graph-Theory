\quest{4.2.1}

\partQuest{a}


\partQuest{b} Seja $G$ um grafo conexo qualquer e $e = vx \in E(G)$, temos que para cada subárvore geradora $T$ de $G$, ou $e \in E(T)$ ou $e \notin E(T)$.
%
Se $e \notin E(T)$, então $T$ é arvore geradora de $G\backslash e$, caso contrário, se $e \in E(T)$, então pela letra (a) temos que $T/e$ é árvore geradora de $G/e$.
%
Logo, $$t(G) = t(G\backslash e) + t(G/e)$$
