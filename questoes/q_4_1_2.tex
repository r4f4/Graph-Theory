\quest{4.1.2}

Vamos provar que a) $\leftrightarrow$ c) e b) $\leftrightarrow$ c).

\partQuest{a}

\partQuest{b}
($\Rightarrow$) Seja $G$ uma floresta com $n-1$ arestas.
%
Como florestas são acíclicas, basta mostrar que $c(G) = 1$ e teremos então uma árvore. 
%
Considere então ${\cal C} = \{c_1, c_2, \ldots, c_k\}$ as componentes conexas de $G$.
%
Cada componente $c_i$ é um grafo conexo, acíclico e com $e(C_i) = v(c_i) - 1$, pois é uma árvore ({\bf Teorema 4.3}).
%
Desta forma, a quantidade de arestas em $G$ é dada por:
\begin{eqnarray}
	e(G) &=& \sum_{c_i \in {\cal C}} e(c_i)      \nonumber \\
	     &=& \sum_{c_i \in {\cal C}} v(c_i) - 1  \nonumber \\
	     &=& \sum_{c_i \in {\cal C}} v(c_i) - \sum_{c_i \in {\cal C}} 1 \nonumber \\
	     &=& n - k \label{eq:componentes}
\end{eqnarray}

Logo, como temos $n-1$ arestas em $G$, então $k = 1$, com o conjunto $|{\cal C}| = 1$ e, portanto, $G$ uma árvore.

($\Leftarrow$) Seja $G$ uma árvore, então, por definição $G$ é floresta. Além disto, $c(G) = 1$ e, utilizando a equação \ref{eq:componentes}, temos que $e(G) = n - 1$ e o resultado segue.
\fimprova

