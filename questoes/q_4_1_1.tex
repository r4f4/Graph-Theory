\quest{4.1.1}

\partQuest{a} {\bf Sol 1)} Provaremos primeiro um resultado mais genérico.
%
Seja $F$ uma floresta com $\Delta(F) = k$, então $F$ tem pelo menos $k$ folhas (Denote por $\# folhas(G)$ o número de folhas na árvore $G$). A prova será por indução forte em $e(F)$.

Se $e(F) = 0$, então o resultado segue trivialmente pois $\# folhas(F) \ge 0$.
%
Caso contrário, seja $P = v_1v_2v_3\ldots v_k$ um caminho maximal em $F$ com $k \ge 2$ (Podemos escolher isto, já que $E(F) \ne \varnothing$).
%
Sabemos então, que $v_1$ e $v_k$ são folhas em $F$, senão ou $P$ não seria maximal ou haveria um ciclo em $F$.
%
Considere então $F' := (V(F),E(F)-E(P))$, que é uma floresta (não contém ciclos), com $e(F') < e(F)$, duas folhas a menos ($v_1$ e $v_k$) e $\Delta(F') \ge \Delta(F) - 2$, pois $\forall v\in V(F)\quad d_F(v) \ge d_{F'}(v) - 2$, já que removemos as arestas de um caminho.
%
Por hipótese de indução, temos que $\# folhas(F') \ge \Delta(F')$.
%
Logo, temos o seguinte: 
\begin{eqnarray}
	\#folhas(F) &=& \# folhas(F') + 2  \nonumber \\
		  &\ge& \Delta(F') + 2 	   \nonumber \\
		  &\ge& \Delta(F) - 2 + 2  \nonumber \\
		    &=& \Delta(F) \nonumber
\end{eqnarray}
\fimprova

Como corolário deste resultado temos a resposta da questão, para o caso em que $F$ é conexa.\\

{\bf Sol 2)} Suponha por absurdo que $T$ seja uma árvore com grau máximo $k$ e
menos que $k$ folhas. Seja $F$ o conjunto das folhas de $T$. Porque $F$ é um
conjunto de folhas, $\partial(F) \le k - 1$. Seja $v$ o vértice tal que
$d(v) = k$. Seja ${\cal P}$  o conjunto de caminhos de $v$ a cada uma das folhas
de $F$. $|{\cal P}| \le k - 1$ pois $v$ pode ter no máximo $k-1$ caminhos usando
arestas incidentes distintas. Seja $e$ a aresta não usada por nenhum caminho em
${\cal P}$ e $u$ o vértice sobre o qual $e$ incide. Sabemos que $u$ não é folha,
pois caso contrário $e$ estaria em ${\cal P}$. Seja $P_{uf}$ o caminho de $u$ até
uma folha $f \in F$. Então $P:=vuP_{uf}P_{fv}$ é um ciclo em $T$, o que é absurdo.\\

\partQuest{b} Um conjunto ${\cal P} = \{P_1, P_2, \ldots, P_k\}$, onde $P_i$, $1 \le i \le k$, é um caminho. Além disto, os caminhos são disjuntos nos vértices, exceto pelo vertice inicial $x$, que é comum a todos.
%
Por exemplo ${\cal P} = \{ xv_1v_2v_3, xz_1z_2z_3z_4, xq_1q_2\}$, que pode ser visto na Figura \ref{graph:estrelaGrande}.
%
Neste caso, temos $d(x) = k$, $d(y) = 2$ (quando $y$ não está no final de algum caminho) e $d(s) = 1$ para os últimos vértices de cada caminho.
%
Além disto, temos $k$ folhas (uma de cada caminho).
%%%%%%%%%%%%%%%%%%%%%%%%%%%%%%%%%%%%%%%%%%%%%%%%%%%%%
%
%
%                       Delta(G) = #folhas(G)
%
%
%%%%%%%%%%%%%%%%%%%%%%%%%%%%%%%%%%%%%%%%%%%%%%%%%%%%%
\begin{figure} [htb]
        \centering
        \begin{postscript}
                \TinyPicture\VCDraw{%
                \begin{VCPicture}{(0,0)(12,12)}
                        \ChgEdgeArrowStyle{-}
                        %%%%%%%%%%%%%%%%%%%%%%%%%%%% vertices %%%%%%%%%%%%%%%%%%%%%%%%%
                        % Vértice em comum
                        \State[x]{(6,6)}{1}
			% Caminho v
			\State[v_1]{(8,6)}{2} \State[v_2]{(10,6)}{3} \State[v_3]{(12,6)}{4}
                        % Caminho q
                        \State[q_1]{(4,4)}{5} \State[q_2]{(2,2)}{6}
			% Caminho z
			\State[z_1]{(4,8)}{7} \State[z_2]{(2,10)}{8} \State[z_3]{(0,12)}{9}
                        %%%%%%%%%%%%%%%%%%%%%%%%%%%% Arestas %%%%%%%%%%%%%%%%%%%%%%%%%
			% caminho v
                        \EdgeL{1}{2}{} \EdgeL{2}{3}{} \EdgeL{3}{4}{}
			% caminho q
                        \EdgeL{1}{5}{} \EdgeL{5}{6}{}
			% caminho z
                        \EdgeL{1}{7}{} \EdgeR{7}{8}{} \EdgeL{8}{9}{}
                \end{VCPicture}}
        \end{postscript}
        \caption {Grafo $G$ com $\Delta(G) = \#folhas(G)$.}
	\label{graph:estrelaGrande}
\end{figure}

