\quest{5.2.5}

\partQuest{a}
$(\Leftarrow)$ Seja $\beta = \{B_1, B_2, \ldots , B_k\}$ o conjunto de blocos
de $G$. Por hipótese temos que $\forall v \in B_i\quad d_{B_i}(v) \equiv 0 (mod$
$2)$, $i = 1, \ldots, k$. Logo, como $d_G(v) = \sum_{i = 1}^{k} d_{B_i}(v)$, 
temos que $d_G(v) \equiv 0 (mod$ $2)$, pois é expresso através da soma de 
parcelas pares.

$(\Rightarrow)$ Seja $G$ um grafo par e $\beta = \{B_1, B_2, \ldots , B_k\}$ 
o seu conjunto de blocos. Sabemos que todo vértice $V$ não separador está 
contido em apenas um bloco $B_i \in \beta$, logo, como $\beta$ é uma 
decomposição de $G$, temos que $d_{B_i}(v) = d_G(v)$ e, portanto $d_{B_i}(v)$ 
é par. Por absurdo, suponha a existência de um bloco $B_i$ com um vértice 
separador $v_1$ de grau ímpar. Então, pelo {\bf corolário 1.2} $B_i$ possui,
pelo menos, mais um vértice ($v_2$) de grau ímpar. Portanto, existe outro
bloco $B_j$ que contém $v_2$ com grau ímpar, pois sabemos que
$\sum_{i = 1}^{k} d_{B_i}(v_2) \equiv 0 (mod$ $2)$. Considere então, $B_j$
o qual contém $v_2$ com grau ímpar. Novamente pelo {\bf corolário 1.2} temos
que existe um outro vértice de grau ímpar $v_3 \in V(B_j)$. Esse argumento 
pode ser usado indefinidamente, até que, pelo princípio da casa dos pombos, 
encontraremos um bloco já visitado na sequência (Spg, $B_i$, $B_j$, \ldots, $B_i$),
e, como $B_i$ é conexo, temos um ciclo $C = v_1B_iv_2B_jv_3\ldots v_kB_iv_1$
contido em vários blocos de $\beta$, o que é uma contradição pelo 
{\bf Teorema 5.3c}. Portanto, não existe vértice separador de grau ímpar, logo
todo bloco de $\beta$ é par.

\fimprova

\partQuest{b}
{\bf Sol 1)} $(\Leftarrow)$ Seja $G$ um grafo e $B = \{B_1, \dotsc, B_k\}$ sua
decomposição em blocos tal que cada bloco é bipartido. Seja $(X_i, Y_i)$ a
bipartição do bloco $B_i$. Sabemos que não podem haver arestas entre dois
blocos pois, pela Proposição 5.3c, eles têm no máximo um vértice em comum.
Considere dois blocos $B_i$ e $B_j$. Se $(X_i,Y_i)$ é bipartição de $B_i$, sem
perda de generalidade suponha que o vértice $v$ comum a $B_i$ e $B_j$ pertença
a $X_i$. Na bipartição de $B_j$, chamaremos de $X_j$ a partição que contém $v$.
Se $v$ não existe, escolha uma partição qualquer como $X_j$. Então, $(X_i \cup
X_j, Y_i \cup Y_j)$ é bipartição de $B_i \cup B_j$. Por que os blocos formam
uma decomposição de $G$, $(\bigcup_{i=1}^{|B|} X_i, \bigcup_{i=1}^{|B|} Y_i)$ é
bipartição de $G$.

$(\Rightarrow)$ Seja $G$ um grafo bipartido e suponha que sua decomposição em
blocos seja tal que existe um bloco $B$ que não é bipartido. Pela Proposição
5.3c sabemos que todo ciclo de $G$ está contido em um bloco. Além disso, pelo
Teorema 4.7, $B$ contém um ciclo ímpar. Mas nesse caso, $G$ também possui um
ciclo ímpar, o que contraria a hipótese de que $G$ é bipartido pelo Teorema
4.7.

